\section{Environment Store}
\subsection{Size dialog}
At start-up, a small GUI is shown that prompts the user to enter the map dimensions of the map has that they want to create. Figure ~\ref{fig:sizeDialog} displays the size dialog.
\begin{figure}[h]
	\includegraphics{NewFeatures/SizeDialog.png}
\caption{The size dialog}
\label{fig:sizeDialog}
\end{figure}
\subsection{Map editor}
After specifying the amount of rows and columns the map should have and clicking on the Start button, the map editor is launched. ~\ref{fig:mapEditor} shows the environment store after launch, with a 5x5 map.
\begin{figure}[h]
	\includegraphics[scale=0.55]{NewFeatures/MapEditor.png}
\caption{The map editor with a 5x5 freshly created map}
\label{fig:mapEditor}
\end{figure}
The map editor has many functionalities. The functionalities are described bit by bit in this documentation.
\subsection{Editing a room}
Right click on any corridor or zone, and the drop down menu in ~\ref{fig:dropMenu} is shown.
\begin{figure}[h]
	\includegraphics[scale=0.55]{NewFeatures/DropDownMenuRoom.png}
\caption{The drop down menu generated when a zone is right clicked}
\label{fig:dropMenu}
\end{figure}
This allows zones to be converted to any of the zones available in the system (charge zones, blockades, rooms, etc.). Figure ~\ref{fig:rooms} shows the different zone appearances in the map editor.
\begin{figure}[h]
	\includegraphics{NewFeatures/DifferentRooms.png}
\caption{The color palette at the bottom of the map editor}
\label{fig:rooms}
\end{figure}
As can be seen in ~\ref{fig:rooms}, the room and drop zone contain a green border. This is to indicate which way the door is faced. This is also editable. When right clicking on a certain room, the drop down menu in ~\ref{fig:menuroom} pops up.
\begin{figure}[h]
	\includegraphics{NewFeatures/DropDownMenuRoom2.png}
\caption{The drop down menu generated when a room is right clicked}
\label{fig:menuroom}
\end{figure}
That drop down menu lists the directions the door can face while still being accessible. In the figure, as it is the top left zone, only doors facing east or south are possible.
\subsection{Editing a sequence}
Located below the map editor, there is a text field which contains the color sequence of the blocks to be added to the map. The same text field is also present in the rooms that aren't drop zones. The text fields are given in ~\ref{fig:seqEditor}.
\begin{figure}[h]
	\includegraphics[scale=0.5]{NewFeatures/SequenceEditor.png}
\caption{The rectangle at the bottom of both the map editor and the rooms}
\label{fig:seqEditor}
\end{figure}
The text field in the rooms can be used to enter colors, which are then used for the blocks that are placed in that room. The text field under the zones can be used to enter the colors that make up the color sequence of blocks to be delivered. To add colors, the color palette is useful.
\subsection{Color palette}
At the bottom of the map editor is the color palette. A larger picture is given in ~\ref{fig:colors}.
\begin{figure}[h]
	\includegraphics{NewFeatures/ColorPalette.png}
\caption{The color palette at the bottom of the map editor}
\label{fig:colors}
\end{figure}
The color palette is used for adding blocks to rooms and the sequence to be completed. When clicking on the box containing either the sequence of the map or the blocks in the room, the color palette is shown. It allows the user to add colors by clicking on the respective colors on the palette, but colors can also be added by entering the respective numbers in the colors (e.g. pressing 1 adds a red block to the room or sequence) and also by entering the first letter of the color (e.g. pressing R adds a red block).
\subsection{File menu}
The file menu does not only contain the standard operations for exporting a created map so that the map can be used later, but also a few other options. The drop down menu is as pictured in ~\ref{fig:dropFile}.
\begin{figure}[h]
	\includegraphics{NewFeatures/DropDownFile.png}
\caption{The drop down menu created when the 'File' menu item is clicked}
\label{fig:dropFile}
\end{figure}
The New, Open..., Save and Save as... options are the standard file operations and need no further explanation. However, as added functionality to the options, it is impossible to save a map not containing either a start- or drop zone, as well as a map not containing a block in its sequence, and saving an unsolvable map will give a warning describing what is wrong. The Exit option also needs no further explanation. The Preview option gives a preview of the created map, as how it would be rendered when used as an actual map for bots. For a non-edited 5x5 map, the preview looks as pictured in ~\ref{fig:preview}.
\begin{figure}[h]
	\includegraphics{NewFeatures/Preview.png}
\caption{The preview of a 5x5 non-edited map, with no blocks in the sequence}
\label{fig:preview}
\end{figure}
\subsection{Tools menu}
The other drop down menu at the top of the map editor is the tools menu. This menu contains options to use several tools. The drop down menu is given in ~\ref{fig:dropTools}.
\begin{figure}[h]
	\includegraphics{NewFeatures/DropDownTools.png}
\caption{The drop down menu created when the 'Tools' menu item is clicked}
\label{fig:dropTools}
\end{figure}
Clicking on the 'Randomize Zones' option creates a random map, based off of a vanilla map with reclassification of rooms to blockades, charge zones or corridors, and with reclassifications of existing corridors to charge zones. The map is always solvable, given that all of the blocks in the sequence are present in the map and no further changes are made to the map. The 'Randomize Blocks' option randomizes the blocks contained in every room according to parameters that the user has to specify. The interface for specifying the parameters is given in ~\ref{fig:menuBlocks}.
\begin{figure}[h]
	\includegraphics[scale=0.6]{NewFeatures/MenuBlocks.png}
\caption{The menu appearing when the randomize blocks option is clicked in the menu}
\label{fig:menuBlocks}
\end{figure}
This interface allows the user to specify which block colors they want to be spawned, and also the maximum amount of blocks in a room. The interface for random sequence generation is slightly different, and is as pictured in ~\ref{fig:menuSeq}.
\begin{figure}[h]
	\includegraphics[scale=0.6]{NewFeatures/MenuSeq.png}
\caption{The menu appearing when the randomize blocks option is clicked in the menu}
\label{fig:menuSeq}
\end{figure}
The differences between this interface and the previous interface are the Randomize button and Result label, that create and display respectively the random sequence, and the apply and cancel button which enter the sequence in the map or cancel the creation of the sequence.