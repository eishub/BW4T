\documentclass[a4paper]{article}
\usepackage{fullpage}
\usepackage{hyperref}
\usepackage{graphicx}
%\usepackage[nottoc,numbib]{tocbibind}
\title{\textbf{Blocks World For Teams} \\ \vspace{0.1cm} \textbf{\Large{Product Vision}} \\ \vspace{1.5cm} \large{by} \\ \vspace{1cm}}
\author{\Large{A Search And Rescue Mission Contextproject}\\\\
\begin{tabular}{lll}
	Katia Asmoredjo & kasmoredjo & 4091760\\
	Joop Au\'{e} & jaue & 4139534 \\
	Wendy Bolier & wbolier & 4133633 \\
	Nick Feddes & nfeddes & 4229770 \\
	Jan Giesenberg & jgiesenberg & 4174720 \\
	Sille Kamoen & skamoen & 1534866 \\
	Sander Lievens & sliebens & 4207750 \\
	Valentine Mairet & vmairet & 4141784 \\
	Arun Malhoe & amalhoe & 4148703 \\
	Tom Peeters & tompeeters & 4176510 \\
	Martin Rogalla & mrogalla & 4173635 \\
	Tim van Rossum & trvanrossum & 4246306 \\
	Joost Rothweiler & jrothweiler & 4246551 \\
	Ruben Starmans & rstarmans & 4141792 \\
	Daniel Swaab & dswaab & 4237455 \\
	Seu Man To & sto & 4064976 \\
	Shirley de Wit & shirleydewit & 4249259 \\
	Calvin Wong Loi Sing & cwongloising & 4076699 \\
	Xander Zonneveld & xzonneveld & 1509608 \\
\end{tabular}
}
\date{	\vspace{1.5cm}Delft University of Technology\\ \vspace{1.5cm}May 2014\\}
\begin{document}
\maketitle

\begin{abstract}
TODO
\end{abstract}
\newpage
\tableofcontents
\newpage
\section{Customer}
\subsection{Usage}
Blocks World For Teams (BW4T) is an environment used to test joint activity. An environment to simulate a Multi Agent System(MAS). A MAS is a system in which not only multiple agents work together as a team to reach a unified goal. The BW4T even has support for the mixed human-agent coordination simulation. The system is used mostly for research purposes as to improve the coordination of human-agent teams for use in for example rescue operations. Thus it will be used to measure the competence of human-agent teams and improve them. Besides that the environment will also be used for educational purposes as to teach students the concept of MAS. This gives the students an insight on how to work with agents and how to use them in an operation.

\subsection{Target customer}
There is not one person who will buy the product. It will be the research groups who buy the product in the name of their university. Several researchers of such a research group will be the main users of the product. They will develop human-agent teams for a variety of scenarios and will need to test them. This environment will be developed just for that. As stated earlier the BW4T environment will also be used for teaching. However this will be less important in development because the students will only learn certain aspects of the overall system. 

In the future the BW4T, or a further developed version of it, might be used by actual rescue teams that have the need of using one or more robots in their mission. They will have the need to extensively test their team before they will actually use it in the field. Before they can start testing their teams with a product like this more developments have to be made to make the system more advanced to being used with real life robots. 

With researchers being the main users of BW4T, it will have to be tailored to their needs. Which will be discussed next.

\section{Customer needs}
The customer has provided us with software designed for team interaction between robots, and possibly even between robots and humans. This software is called Block Worlds For Teams (BW4T), and it is part of the study of Joint Activity. As participants in deep research among human robot interaction, the customer wants to make use of our product for better research capability and eventually, usage in the real world.

There are two main customer needs we need to fulfill: the first one being the ease of use of the BW4T software. The customer wants to be able to manipulate the software without having to worry about consequences. For instance, if they remove some of the functionalities, it will not have any effect on the rest of the software. The customer also wants to be able to translate agents’ orders to the robots that will be used for Search and Rescue missions. In order to achieve this, we need to make good use of software called Repast and combine it with GOAL as smoothly as possible. For the study of Join Activity, it is essential to be able to run simulations with robots who are assigned a certain amount of tasks. But to perform simulations that will give good and valid results, it is mandatory to make it as realistic as possible. That is exactly what the customer needs: a realistic environment with a realistic robot behavior, so they can observe and draw consistent conclusions from the simulations.

For the ease of use of the product, the customer has two main interfaces provided: a scenario GUI and human GUI, in order to run simulations properly. With the scenario GUI, the customer can customize the environment they want their robots to perform in. They can also add certain features to their robots through that same UI; features such as wheels, extra speed, feet, etc… Many scenarios can be thought up, for instance: an avalanche occurred and people need to be rescued as soon as possible: how great will the robots perform? As the program runs, the customer can observe and trust their robots’ performance. With the human GUI, a human can take control of the situation and run in the environment. The customer wants to research human robot interaction; this is the time and place to do it. To facilitate joint activity with a human, we have the concept of e-Partner available for the customer. A human can take control of the robots with the e-Partner feature. The e-Partner can also guide the human through the environment and can facilitate certain tasks.

The last feature provided for the customer has already been mentioned above: realism. It is primarily important to create environments and robot capabilities as realistic as possible; otherwise research results would be flawed. The customer will eventually use the product on real Search and Rescue missions, with real robots in a real environment. Realism and accuracy of the simulations will provide the certainty of the simulation results in real life. Given the realism of the product, the customer can fully develop a robot team that will successfully complete the mission. 

\section{Comparison}
The product we are developping is an upgraded version of the existing BW4T simulation environment. It is extended in a few ways. First, the bots can see each other when there are no obstacles in between them and they can not walk over each other. If they bump into each other, health loss may occur. This feature makes the simulation more realistic. Second, there is an environment store where maps can be randomly generated, manually created and saved. This makes it more convenient to test on random senario's and because of the option to manually create maps, more realistic environments can be made to test on. Another new feature is the e-Partner. This feature can be used in various ways to assist the human, which makes more diverse simulations possible.

This product is similar to Gamebots: A 3D Virtual World Test-Bed For Multi-Agent Research \cite{adobbati2001gamebots} in the aspect of multiple environments creation and the human-agent collaboration. But what makes our product stand out is that there are no restrictions to the number of human and agent teams.

\section{Time Frame \& Budget}
The project of improving the current Blocks World For Teams environment will take place in around six to eight weeks. This block of time will be split up in separate weeks. With one to two weeks in the beginning to set and start up the project it will be assured that all aspects covered will be accounted for in the following weeks. This will ensure that the rest of the process will go as smooth as possible. This is necessary because the biggest part of the project, which consists of 6-7 weeks, will be managed in the Scrum framework. This means every week a scrum has to take place. This results in a tight schedule every week. This is because at the end of the week a new working and tested prototype will be delivered. At the end of these eight weeks a final product will be delivered.

Every project has its own expenses which are paid out of a budget. We as a group of students working on a research project do not have an actual budget to pay certain things like server costs from. But since we work for a university they can possibly provide us if we need anything. Thus keeping the cost low or even keeping it to zero.


\clearpage
\bibliographystyle{plain}
\bibliography{product-vision}
\addcontentsline{toc}{section}{References}

\end{document}