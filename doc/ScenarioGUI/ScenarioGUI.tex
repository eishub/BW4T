\documentclass[a4paper]{article}
\usepackage{fullpage}
\usepackage{hyperref}
\usepackage{graphicx}
\title{Scenario GUI}
\author{Group 3 of the A Search And Rescue Context Project}
\date{}
\begin{document}
\maketitle
\newpage

\tableofcontents
\newpage

\section{Introduction}
This document describes how to make use of the Scenario GUI. The GUI exists of two panels. The first one is the Configuration Panel, which is used to create, open and save configuration files. The second panel is the Bots Panel, where a list of the bots is being displayed. Here you can create, modify, rename, duplicate and delete bots. On the top of this panel is indicated how many bots, e-partners and humans are in the bot list.

\section{Configuration Panel}
\subsection{Create configuration file}
\begin{enumerate}
	\item To create a new configuration file, click on File -> New.
	\item Fill in the Client IP and the Client Port.
	\item Fill in the Server IP and the Server Port.
	\item Decide whether you want to use GOAL. Check the yes or no box.
	\item Decide whether you want to launch the GUI. Check the yes or no box.
	\item Click on Open File in the Agent class section to select an agent class file.
	\item Click on the Open File in the Map file section to select a map file.
\end{enumerate}

\subsection{Save configuration file}
\begin{enumerate}
	\item To save the newly created configuration file with the form filled out: click on File -> Save.
	\item Select the a folder where you want to save the file and click Save.
\end{enumerate}

\subsection{Open configuration file}
\begin{enumerate}
	\item To open a configuration file, click on File -> Open.
	\item Select the file you want to open and click Open.
\end{enumerate}

\section{Bots Panel}
\subsection{Create new bot}
\begin{enumerate}
	\item To create a new bot, click on New bot. You will be redirected to the Bot GUI.
\end{enumerate}

\subsection{Modify bot}
\begin{enumerate}
	\item To modify a bot, select the bot you want to modify in the bot list and click on Modify bot. You will be redirected to the Bot GUI.
\end{enumerate}

\subsection{Rename bot}
\begin{enumerate}
	\item To rename a bot, select the bot you want to rename in the bot list.
	\item Type the new name in the box above the Rename bot button and click on Rename bot.
\end{enumerate}

\subsection{Duplicate bot}
\begin{enumerate}
	\item To duplicate a bot, select the bot you want to duplicate in the bot list.
	\item Type in the number of how many times the bot needs to be duplicated in the box above the Duplicate bot button and click on Duplicate bot.
\end{enumerate}

\subsection{Delete bot}
\begin{enumerate}
	\item To delete a bot, select the bot you want to delete and click on Delete bot.
\end{enumerate}
\end{document}